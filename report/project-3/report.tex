\documentclass{article}
\usepackage{geometry}
\usepackage{float}
\usepackage[T1]{fontenc}
\usepackage[polish]{babel}
\usepackage[utf8]{inputenc}
\usepackage{graphicx}
\graphicspath{ {./images/} }

\geometry{
 a4paper,
 total={170mm,257mm},
 left=20mm,
 top=20mm,
}

\title{Analiza i przetwarzanie dźwięku - Projekt 3}
\date{\today}
\author{Mateusz Śliwakowski}

\begin{document}
  \pagenumbering{gobble}
  \maketitle
  \pagenumbering{arabic}
  
\section{Treść zadania}
Celem zadania było zaimplementowanie programu, który wczytuje plik audio a następnie przeprowadza jego analizę w dziedzinie częstotliwości na poziomie ramki. Należało zbadać cztery parametry - \textit{Volume, Frequency Centroid, Effective Bandwidth} oraz \textit{Band Energy}.

\section{Opis aplikacji}

Aplikację zdecydowałem się wykonać w języku \textit{C\#} w środowisku \textit{WinForms}. Jako, że w zadaniu można było wykorzystać znaczną część implementacji z projektu 2 zdecydowałem, że zamiast tworzyć nowe rozwiązanie, projekt zrealizuję poprzez rozwój poprzedniego. Prace rozpocząłęm od dodania zakładek z nowymi parametrami.

\begin{figure}[H]
\includegraphics[width=6in]{scr1.png}
\centering
\caption{Interfejs użytkownika}
\label{fig:interface}
\end{figure}

Użyte technologie pozostały bez zmian. Do obsługi plików audio użyłem biblioteki \textit{NAudio}, do rysowania wykresów \textit{OxyPlot}, a do obliczania FFT \textit{MathNet Numerics}.

\section{Zaimplementowane metody}
\subsection{Wstęp}
Wszystkie zaimplementowane metody opierały się na analizowaniu sygnału audio w dziedzinie częstotliwości na poziomie ramki. Wejściowy sygnał był dzielony na ramki o zadanej długości (zachodziły one na siebie zgodnie z wartością parametru \textit{overlap}).

\subsection{Volume}
Głośność w tym opracowaniu wyznaczana była w inny sposób, niż w projekcie pierwszym. Dla danej ramki określona była następującym wzorem:
$$Vol(n) = \frac{1}{N}\sum^{N-1}_{k=0}S_n^2(k),$$
gdzie $S_n$ to wynik transformaty Fouriera dla ramki o indeksie n, a $N$ to długość ramki.

\subsection{Frequency Centroid}
Kolejnym badanym parametrem był \textit{Frequency Centroid}. Określa on gdzie zlokalizowany jest środek ciężkości dla spektogramu. Dla odbiorcy odpowiada wrażeniu jasności dźwięku. Wyraża się wzorem:
$$FC(n)=\frac{\int_{0}^{\infty} \omega S_n(\omega)d\omega}{\int_0^\infty S_n(\omega)d\omega},$$
gdzie $\omega$ to częstotliwość. Ze względu na fakt dyskretyzacji danych, całka w implementacji została zamieniona na sumę.

\subsection{Effective Bandwidth}
Używając obliczonego \textit{Frequency Centroid} można policzyć parametr bezpośrednio z nim związany, czyli \textit{Effective Bandwidth}. Określa on odchylenie standardowe częstosliwości, przez co dobrze określa szerokość pasma. Wyraża się wzorem:
$$BW^2(n)=\frac{\int_0^\infty(\omega-FC(n))^2S_n^2(\omega)d\omega}{\int_0^\infty(\omega)d\omega}.$$
Również w tym przypadku, konieczne było zamienienie całki na sumę w implementacji.

\subsection{Band Energy}
Ostatnim parametrem było \textit{Band Energy}, obliczane w przedziale częstotliwości od $f_0$ do $f_1$. Określa on energię w spektrum sygnału i określony jest wzorem:
$$BE_{[f_0,f_1]}(t)=\frac{\int_{f_0}^{f_1}S_t^2(f)df}{\int w(\tau)d\tau},$$
gdzie w jest oknem analizy.

\section{Wyniki działania programu}

\section{Wnioski}
\subsection{Implementacja}
Posiadając podłoże do pracy w postaci projektu drugiego, implementacja przebiegła szybko i sprawnie. Można było się oprzeć na kluczowych funkcjonalnościach takich jak podział sygnału na ramki, obliczanie transformaty Fouriera, czy rysowanie wykresów. Trudność sprawić mogło dostosowanie wzorów z całkami do obliczeń dyskretnych, lecz nie były one na tyle skomplikowane, aby okazało się to dużą przeszkodą.

\subsection{Wyniki analizy audio}


\begin{figure}[b]
\centering
\includegraphics[width=5in]{bottom.png}
\end{figure}

\end{document}

